\section{Zadanie 3}
Należy udowodnić, że transmitancja obliczona na podstawie pierwszego modelu w przestrzeni stanu jest taka sama, jak obliczona na podstawie drugiego modelu.
Ponieważ transmitancja
\[
 G(z)=C(zI-A)^{-1}B+D
\]
to można porównać oba te modele ustawiając $z$ jako zmienną symboliczną.

Obliczmy transmitancje od pierwszego i drugiego modelu.
\begin{minted}{matlab}
syms z;
trans1 = Cd * inv(z * eye(3) - Ad) * Bd + Dd;
trans2 = Cd2 * inv(z * eye(3) -Ad2) * Bd2 + Dd2;
\end{minted}
Używając funkcji porównującej \mintinline{matlab}{isequal(trans1,trans2)} otrzymujemy informację o logicznej wartości prawdziwej.
\begin{minted}{matlab}
ans =

  logical

   1
\end{minted}
 
Co pokazuje, że transmitancje są sobie równe. Alternatywnie upraszczając wzory używając \mintinline{matlab}{simplify(trans1)} i \mintinline{matlab}{simplify(trans2)}
możemy je naocznie porównać po wypisaniu na ekran.
\begin{minted}[breaklines]{matlab}
trans1 =
 
(3*(6261984766567268*z^2 - 6946490535369048*z + 1828337718241467))/(2*(36028797018963968*z^3 - 116297958657721184*z^2 + 52215069291327552*z - 6260866135760997))
 
 
trans2 =
 
(3*(6261984766567268*z^2 - 6946490535369048*z + 1828337718241467))/(2*(36028797018963968*z^3 - 116297958657721184*z^2 + 52215069291327552*z - 6260866135760997))
\end{minted}

