\section{Zadanie 1}
Celem zadania jest wyznaczenie transmitancji dyskretnej od transmitancji ciągłej:
\[
 G(s)=\frac{(s + 2)(s + 3)}{(s - 4)(s + 5)(s + 6)}=\frac{s^2 + 5 s + 6}{s^3 + 7 s^2 - 14 s - 120}
\]
z okresem próbkowania $T=0,25$.

Do wykonania tego można użyć programu MatLab. Najpierw należy ustawić zmienne i obliczyć za pomocą \mintinline{matlab}{c2dm}.
Ekstrapolator zerowego rzędu osiągamy ustawiając \mintinline{matlab}{'zoh'}, co oznacza, że wartość próbki jest podtrzymywana w czasie jej trwania.
\begin{minted}{matlab}
T = 0.25;
licz = [0 1 5 6];
mian = [1 7 -14 -120];
[liczdys,miandys] = c2dm(licz,mian,T,'zoh');
\end{minted}
Wynikiem są:
\begin{minted}{matlab}
liczdys =

         0    0.2607   -0.2892    0.0761
         
miandys =

    1.0000   -3.2279    1.4493   -0.1738
\end{minted}

Co się przekłada na:
\[
 G(z)=\frac{0,26z^2 - 0,29z + 0,08}{z^3 - 3,23z^2 + 1,45z - 0,17}
\]

Zera transmitancji ciągłej można wyliczyć przyrównując licznik do zera, podobnie bieguny przyrównując mianownik:

\[
\left\{
\begin{array}{l}
	s_{z1}=-2	\\
	s_{z2}=-3	\\
	s_{b1}=4	\\
	s_{b2}=-5	\\
	s_{b3}=-6	\\
\end{array}
\right.
\]

Używając funkcji \mintinline{matlab}{roots} łatwo obliczamy także zera i bieguny transmitancji dyskretnej:

\[
\left\{
\begin{array}{l}
	z_{z1}=0,68	\\
	z_{z2}=0,43	\\
	z_{b1}=2,72	\\
	z_{b2}=0,29	\\
	z_{b3}=0,22	\\
\end{array}
\right.
\]
