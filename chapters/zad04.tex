\section{Zadanie 4} 
Stabilność układu zależy od umiejscowienia wartości własnych macierzy $A$, bądź biegunów transmitancji na płaszczyźnie zespolonej.
Jeśli jakakolwiek z tych wartości ma dodatnią część rzeczywistą, to obiekt jest niestabilny. 

Korzystając z wyznaczonych wcześniej biegunów możemy wywnioskować, że obiekt jest \emph{niestabilny}, bowiem jeden z biegunów $s_{b1}=4$ ma dodatnią część rzeczywistą.
Podobnie dla transmitancji zespolonej, jednak tutaj warunkiem jest zawieranie się bieguna w kole jednostkowym. To, czego $z_{b1} = 2,72$ nie spełnia.
Odpowiedź układu na skok jednostkowy nie stabilizuje się, tylko ucieka do nieskończoności.

Regulator jest potrzebny, aby zarówno sterować obiektem, ale także aby stabilizować go. Pełni zatem dwie funkcje.
Bez regulatora obiekt nigdy nie osiągnąłby stanu ustalonego.